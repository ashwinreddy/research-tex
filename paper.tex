\documentclass[12pt]{article}
\title{An Investigation of Deep Reinforcement Learning based Algorithms for Robotic Control and Manipulation Tasks:\\ A Robot Learning Gym}
\author{Ashwin Reddy  \\
	The Harker School
	}
\begin{document}
\maketitle

\begin{abstract}
    Heavy experimentation is needed in order to apply deep learning to robotics control tasks. There are a handful of environments that allow for such experimentation in the broader field of reinforcement learning. There are also simulators that allow for basic task environments to be created. However, these simulators and environments are mostly used for simple games and robotics control tasks. I propose a toolkit specifically targeted for robot reinforcement learning tasks, especially ones that use deep learning.
\end{abstract}


\section{Introduction}
\paragraph{}
\section{Related Work}
OpenAI Gym is a platform that seeks to consolidate various tasks \cite{1606.01540}. However, only some of these tasks are close to the kinds of visuomotor tasks needed to simulate robotics skills.

\section{Robot Learning Gym}
Using MuJoCo (CITE THIS!)

Build a catalog of agents, tasks, and robots.

GPS has produced impressive results \cite{lfda-e2e-15}.
GPS has a code base \cite{fzftm-gpsi-16}
Transfer learning algorithms:
A specific example here is modular nets \cite{DBLP:journals/corr/DevinGDAL16}

RLG is a package developed in Python that has been published online with documentation.

The package is available at \texttt{https://www.github.com/ashwinreddy/rlg}
\section{Discussion and Future Work}



\bibliographystyle{apalike}
\bibliography{paper}
% \[u=vwxz\]
\end{document}
